\part{AMS Detector}
AMS-02 is a particle detector mounted on the upper Payload Attach Point(S3) on the main truss of ISS, which orbits the Earth at an altitude of about 300 km. AMS-02 studies with an unprecedented accuracy of the cosmic ray particles, such as $e^{\pm}$, $\gamma$, $\bar{p}$, $\bar{D}$ and nuclei from H to Fe.\\
The objectives of AMS-02 include search for primordial antimatter and evidence of dark matter by measuring $\bar{He}$, anti-proton and positron. By measuring the ratio $\bar{p}/p$ and $B/C$, AMS can refine the propagation models.
A particle is identified by its properties such as charge and energy. The design of AMS detector assures redundant and independent measurements of such properties.
\chapter{AMS sub-detectors}
AMS-02 has five sub-detectors. They are 
\section{Transition Radiation Detector}
\section{Time of Flight}
\section{Silicon Tracker and Magnetic Field}
\section{Ring Imaging Cherenkov Detector}
\section{Electromagnetic Calorimeter}
\section{Particle identification}

\chapter{Performance of ECAL}

\chapter{AMS trigger system}
Among the sub-detectors presented above, ToF and ECAL have independent trigger systems~\cite{triggerTwiki}. They can generate fast triggers for precise timing measurement.The time required for fast trigger decision is 40ns. The fast triggers FTC and FTZ, respectively designed for charged particle and big Z particle, are generated by ToF; the fast trigger FTE, specially designed for neutral particle detection, is generated by ECAL.\\
An event is registered if the particle satisfies one of the triggers. For ISS data, the trigger rate is about 500 events per second. And after 1 year's operation, 17 billion events are registered by AMS-02.